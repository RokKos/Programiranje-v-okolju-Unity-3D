{\color{indiagreen}\subsection{Objektno programiranje}}
Kot nam že samo ime pove temelji objektno programiranje na objektih, ki programiranje bolj pribljižajo človeku. Na tem principu deluje tudi Unity s svojim jezikom C\#. V vsaki igri kot tudi v realnem svetu imamo stvari s podobnimi lastnostmi ali pa celo z enakimi atributi. Tukaj v igro vstopijo objekti. Ti nam pomagajo specificirati atribute določenega predmeta. Vzemimo npr. svetilko. Svetilka ima žarnico, baterijo in stikalo za vklop in izklop. Pozna tudi metodo vklop in izklop. Lahko bi vzeli tudi drugi klas npr. zrcalo, ki bi spet imelo drugačne atribute. Ampak glavna ideja za tem je, da imamo lahko sedaj v naši igri poljubno mnogo instanc teh svetilk in zrcal in zato nismo rabili pisati code za vsakega posebej. Lahko bi rekli, da imamo glavni klas in iz njega samo kopiramo ven nove objekte. Pri razredih pa pridemo tudi do dedovanja, dostopnosti (public, protected, private) in polimorfizma.