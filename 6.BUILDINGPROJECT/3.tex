{\color{indiagreen}\subsection{Unity Remote 4}}
Ker je konstantno grajenje in nameščanje aplikacij zamudno in počasno opravilo, so pri Unity-ju naredili aplikacijo Unity Remote 4. To je zelo uporabna in skoraj nujno potrebna aplikacija za razvijanje na telefonih, kar prihrani ure in ure čakanja. Ta aplikacija se dobi na Google Play Storu in App Storu. Pomaga pri simuliranju:
\begin{itemize}
	\item Dotika
	\item Accelerometer-a
	\item Gyroscopa
	\item Kamere
	\item Kompasa
	\item GPS-a
\end{itemize}
Aplikacija deluje tako, da se preko USB kabla poveže z računalnikom in ko gremo v play mode v Unity urejevalniku pošilja sliko na zaslonu na telefonu. Nazaj pa na računalnik pošilja podatke o zgoraj navedenih funkcijah. Deluje kot nek emulator za telefone. Seveda pa ni še čisto popolna, včasih ima težave z povezavo in resolucija, ki je prikazana na telefonu ni dejanska končna resolucija. Vse razen tega pa dela zelo dobro.