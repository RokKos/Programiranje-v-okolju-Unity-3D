{\color{indiagreen}\subsection{Plugini}}
V Unity-ju si lahko pomagamo z različni skriptami, ki jih ustvarimo izven Unity-ja. Tem skriptam pravimo vtičniki (plugins). Te nam lahko pomagajo pri samem urejanju projekta, olajševanju in avtomatiziranju nekaterih stvari ali pa nam dodeljuje kakšne funkcije za prav posebno platformo. Prvim pravimo Managed plugins, drugim pa Native plugins \cite{manual}. Managed plugine ponavadi pišemo v C\#, saj uporavljajo samo kjižnico .NET. Te vtičniki se prevedejo v dinamične knjižnice (dynamical linked library) oz. DLL, ki jih potem vključimo v projekt. Native plugini pa so napisani v C, C++, Objective-C in ostalih jezikih. To pomeni, da damo možnost Unity, da lahko npr. kliče kodo Java ali C++. Seveda je glavna prednost tega, da napišemo svoje knjižnice za določeno platformo npr. za IOS svojo in za Android svojo. 