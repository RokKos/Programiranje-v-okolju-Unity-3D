{\color{indiagreen}\subsection{Definicija}}
Shaderji so skripte, ki nam povejo, kako naj se vsak piksel na zaslonu zrendera. To seveda ni samo od tega kako so napisani shaderji ampak tudi od tega kakšne materiale in teksture uporabljamo. V shaderjih so zapisani algoritmi, ki uporabljajo vektorje za svetlost in barvo in s tem povejo kako naj se obarva piksel. Z verzijo Unity 5 je prišel ven tudi njihov Standart Shader, ki je zelo uporaben in se lahko zelo prilagaja. Ampak v tem projektu bom predstavil, kako napisati lasten shader. 