\documentclass[a4paper,oneside,12pt]{article} %clanek/poročilo, ki je formata A4, enostransko, s pisavo 12

%poskrbi za prikak šumnikov
\usepackage[slovene]{babel} 
\usepackage[utf8]{inputenc}

%prikaz posebnih znakov
\usepackage[T1]{fontenc}
\usepackage{arev} %font package

%za urlje, vsi urlji so enake barve da je bolj pregledno(uporaba \url{pot} ali \href{pot}{ime})
\usepackage[bookmarks, colorlinks=true, %
linkcolor=black, anchorcolor=black, citecolor=black, filecolor=black,%
menucolor=black, runcolor=black, urlcolor=blue%
]{hyperref}

%za includanje slik
\usepackage{graphicx}
\graphicspath{ {./Slike/} }

%importanje barv
\usepackage[usenames]{color}

%deiniranje barv, barve iz strani http://latexcolor.com/
\definecolor{indiagreen}{rgb}{0.07, 0.53, 0.03}
\definecolor{internationalorange}{rgb}{1.0, 0.31, 0.0}
\definecolor{alizarin}{rgb}{0.82, 0.1, 0.26}
\definecolor{bostonuniversityred}{rgb}{0.8, 0.0, 0.0}

%za pisanje matematicnih formul(reqno pomeni, da gre zaporedna cifra na desno stran)
\usepackage[reqno]{amsmath}

%dodatni matematicni simboli
\usepackage{amssymb}

%doda dodatne moznosti za okolje enumerate
\usepackage{enumerate}

%doda dodatne moznosti za okolje array
\usepackage{array}

%nastavi pozicijo strani, vsi margini
\usepackage[
    paper=a4paper,
    top = 2.0cm,
    bottom = 2.9cm,
    left = 3.0cm,
    right = 2.5cm,
    ]{geometry}

%za importanje datotek in lažje managiranje projekta
\usepackage{import}

%za pravilno postavljanje vejic v matematicnih izrazih
\usepackage{icomma}

%za lepše prikazovanje enot
\usepackage{units}

%za barvanje code oz higliting sintax
\usepackage{minted}

\usefont{T1}{cmr}{b}{n}
%\renewcommand{\familydefault}{\sfdefault} %drugace font

\usepackage[makeroom]{cancel} % za crtanje texta
\usepackage{tikz}
\usepackage{pgfplots} %grafi
\pgfplotsset{
    standard/.style={
        every axis x label/.style={at={(current axis.right of origin)},anchor=north west}, %za dajanje labelov na x osi (yticklabels = {0, $v_0$})
        every axis y label/.style={at={(current axis.above origin)},anchor=north east} %za dajanje labelov na y osi
    }
}
\usepgfplotslibrary{fillbetween} %risanje po grafu
\usepackage{empheq} % za okvirjanje enačb
\usepackage{gensymb} %za stopinj celzije

%izdelava footerja in headerja 
\usepackage{fancyhdr}
\usepackage{lastpage}

%za vec slik v enem
\usepackage{subcaption}

\pagestyle{fancy}
\fancyhf{}
 
\lhead{KOS, Rok, Programiranje v okolju Unity 3D, Proj. nal., Ljubljana, Gimnazija Vič, 2016}
\rhead{Stran \thepage \hspace{1pt} od \pageref{LastPage}}

%spreminjanje naslovov in podnaslov
\usepackage{titlesec}
\titleformat*{\section}{\LARGE\bfseries}
\titleformat*{\subsection}{\Large\bfseries}
\titleformat*{\subsubsection}{\large\bfseries}

%za citiranje
\usepackage[autostyle]{csquotes}
%za vire in literaturo
\usepackage[
    backend=biber, %backend za bibliografijo
	style=alphabetic, %stil bibliografije
	sorting=ynt %sortira po letu, avtorju in naslovu
]{biblatex}
\addbibresource{bibliografija.bib} %file kjer so navedeni viri in literatur

%za includanje izvorne code
\newmintedfile[htmlsource]{html}{linenos=true, mathescape, xleftmargin=0.7cm,
               fontsize=\small,baselinestretch=0.9, breaklines=true, breakanywhere}

\newmintedfile[csharpsource]{csharp}{linenos=true, mathescape, xleftmargin=0.7cm,
               fontsize=\small,baselinestretch=1.0, breaklines=true, breakanywhere}

\newmintedfile[jsonsource]{json}{linenos=true, mathescape, xleftmargin=0.7cm,
               fontsize=\small,baselinestretch=0.9, breaklines=true, breakanywhere}

\newmintedfile[bashsource]{bash}{bgcolor=black,formatcom=\color{white}, linenos=false, mathescape, xleftmargin=0.7cm,
               fontsize=\small,baselinestretch=0.9, breaklines=true, breakanywhere}

%naredi naslov, avtorja in datum
\title{Programiranje v okolju Unity 3D}
\author{Rok Kos} 
\date{\today}


\renewcommand{\abstractname}{Izvleček/Povzetek} %sprememba imena 

\begin{document}
	
	\begin{titlepage}
		\centering
		\includegraphics[width=15cm, height=15cm,keepaspectratio=true]{Logo.png}\par
		{\tiny http://projekti.gimvic.org/2015/2d/gimnazijavic/logo.png} \par\vspace{1cm}
		{\scshape\LARGE Gimnazija Vič \par}
		Tržaška cesta 72, 1000 LJubljana Slovenija \par
		\vspace{1cm}
		{\scshape\Large Seminarska naloga pri predmetu informatika\par}
		\vspace{1.5cm}
		{\huge\bfseries Programiranje v okolju Unity 3D\par}
		\vspace{2cm}
		Avtor: \par
		{\Large\itshape Rok Kos \par}
		\vfill
		Mentor:\par
		prof.~Klemen \textsc{Bajec}

		\vfill

	% Bottom of the page
		{\large \today\par}
	\end{titlepage}

	%\maketitle

	\newpage %nova stran
	%prikaze vse kategorije
	\tableofcontents

	%prikaze kazalo slik
	\listoffigures
	
	\newpage

	\begin{abstract}
		V svoji seminarski nalogi bom predstavil rokovanje z programskim okoljem Unity 3D.\\
		V dan današnjem svetu je v hitrost razvijanja aplikacij ali prototipov zelo pospešena. Predvsem igre in pa tudi ostale aplikacije nastajajo po potrebi in imajo zelo omejen čas za izdelavo. Poskušal vam bom predstaviti zelo kompleksno razvijalsko orodje, ki je obenem tudi zelo močen game engine. Kot primer uporabe bom vzel svojo aplikacijo, ki sem jo ustvaril z Unity-jem, ki predstavlja vodiča po Gimnaziji Vič.\\

		Ključne besede: Unity 3D, game engine, prototip, razvoj aplikacij.
	\end{abstract}
	
	\newpage

	{\color{internationalorange}\section*{UVOD}}
	Že od vedno me je zanimalo ustvarjanje in igranje iger. Preden sem spoznal Unity sem pisal v C++ in Python-u. Ampak moje igre so zgledale več ali manj bolj tekstovne igre v terminalu ali pa na 2D ploskvi. Ko sem spoznal Unity sem videl, programiranje in izdelava 3D, kaj šele 2D iger sploh ni tako zahtevna. Seveda se z Unity-jem niso rešile vse težave pri izdelavi iger temveč so se odprle nove možnosti za raziskovanje tega zelo močnega orodja. V tej seminarski nalogi vam bom probal predstaviti samo vrh ledenika pod katerim se skriva zelo veliko še neodkritega. Tudi sam še nisem popolnoma obvladal tega orodja ampak se počutim že kar domačega.\\
	Unity uporablja zelo veliko uspešnih podjetji(npr. Outfit7) in zelo veliko iger je bilo izdanih s tem orodjem. To orodje povezuje skupaj tri veje skupaj. Te veje so risanje, 3D modeliranje in programiranje(game design). V moji seminarski nalogi bom poskušal predstaviti predvsem slednjega. 


	\newpage	

	{\color{internationalorange}\section{NAMESTITEV OKOLJA}}
	\foreach \n in {1,...,3}{
		\subimport{1.SET_UP/}{\n.tex}
	}

	{\color{internationalorange}\section{SHADERS}}
	\foreach \n in {1,...,2}{
		\subimport{2.SHADERS/}{\n.tex}
	}

	{\color{internationalorange}\section{RAZREDI}}
	\foreach \n in {1,...,2}{
		\subimport{3.CLASSES/}{\n.tex}
	}
	{\color{internationalorange}\section{PREFABS}}
	\foreach \n in {1,...,2}{
		\subimport{4.PREFABS/}{\n.tex}
	}

	{\color{internationalorange}\section{VKLJUČEVANJE}}
	\foreach \n in {1,...,3}{
		\subimport{5.IMPORTING/}{\n.tex}
	}
	{\color{internationalorange}\section{GRAJENJE PROJEKTA}}
	\foreach \n in {1,...,3}{
		\subimport{6.BUILDINGPROJECT/}{\n.tex}
	}
	{\color{internationalorange}\section{VERSION CONTROL}}
	\foreach \n in {1,...,2}{
		\subimport{7.VERSIONCONTROL/}{\n.tex}
	}

	{\color{internationalorange}\section{ZAKLJUČEK}}
	Ko enkrat zgradimo projekt in dobimo apk ali exe datoteko potem se lahko najprej z njo igramo. Ampak to še ni dovolj, to aplikacijo je treba deliti tudi z ostalimi. Saj je namen aplikacij(programov) je da človeku olajšajo življenje in avtomatizirajo posamezne procese. Tako da naslednji korak bi bil uploadati aplikacijo na Google Play Store in pokazati aplikacijo ostalim in zbrati čimveč koristnih povratnih informacij. Potem pa aplikacijo glede na informacije popraviti in prilagoditi.\\
	Po mojem mnenju ima Unity zelo uspešno prihodnost, saj ima za sabo veliko skupnost in razvijalce, ki konstantno prenavljajo to orodje in dodajajo nove funkcije. Mislim tudi, da bodo kmalu tudi večja podjetja raje kupila licence za takšna orodja in ne razvijala svojih orodji, ker se jim po eni strani ne splača po drugi strani pa se večina razvijalcev začne učiti na takih orodjih. Kar je tudi velika prednost da je brezplačno in dostopno vsem od hobi razvijalcev do velikih podjetij. 

	\newpage
	
	%\printbibliography
	\printbibliography[
		heading=bibintoc, %pove da naj bo tudi v categorijah
		title={Viri in literatura} %naslov bibliografije
	]
\end{document}